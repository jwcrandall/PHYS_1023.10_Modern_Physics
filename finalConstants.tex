\documentclass[6pt, oneside]{article}   	% use "amsart" instead of "article" for AMSLaTeX format
\usepackage[margin=0.25in]{geometry}                		% See geometry.pdf to learn the layout options. There are lots.
\geometry{letterpaper}                   		% ... or a4paper or a5paper or ... 
%\geometry{landscape}                		% Activate for for rotated page geometry
%\usepackage[parfill]{parskip}    		% Activate to begin paragraphs with an empty line rather than an indent
\usepackage{graphicx}				% Use pdf, png, jpg, or eps§ with pdflatex; use eps in DVI mode
								% TeX will automatically convert eps --> pdf in pdflatex		


\usepackage[utf8]{inputenc}
\usepackage[english]{babel}
 
\usepackage[usenames, dvipsnames]{color}

\usepackage{color,soul}
\usepackage{multicol}


\title{Modern Physics Constants}
\author{Joseph Crandall}
%\date{}							% Activate to display a given date or no date

\begin{document}
%\begin{multicols}{1}
\colorbox{red}{Joe Crandall's Grade A ModernPhysics constants Sheet, covers special relativity and basic quantum mechanics}

%\maketitle
%\section{}
%\subsection{}



\hl{Quantum Theory of Light}\\
$e_{total}$ = is the power per unit area per unit frequency emitted by the black body\\
a = black body coefficient between 0 and 1, 1 is ideal\\
Bragg equation\\
n = the order of the intensity maximum\\
$\lambda$ = the x-ray wavelength\\
$\theta$  = angle of intensity maximum measured form plane A\\
d = the spacing between planes\\
X-ray photon emission\\ 
V = x-ray tube voltage\\
e = elementary charge\\
\hl{Particle Nature of Matter}\\
Rydberg Constant $R_{\infty}= \frac{m_e e^4}{8\epsilon_0^2 h^3 c} = 1.097x10^{7} m^{-1}$\\
n =  positive integer values\\
$\lambda$ = wavelength of the emitted or absorbed light\\
coulomb constant  $k_e = \frac{1}{4\pi \epsilon_0} = 8.987x10^9 Nm^2C^{-2}$\\ 

\colorbox{BurntOrange}{The Particle Nature of Matter}
$R = 1.0973732x10^7 m^{-1}$ Rydberg constant
\hl{$K_B = 1.38064852x10^{-23}m^2 Kg s^-2 K^-1$ Boltzman Constant}
$h = 6.626x10^{-34}Js$ Plank Constant 
\hl{$c = 2.998x10^{8} ms^{-2}$ Speed of light}

\colorbox{BurntOrange}{The Wave Nature of Matter}
$m_e = 9.10938356x10^{-31}Kg$ Mass of an Electron
$\hbar = \frac{h}{2\pi} = 1.054571x10^{-34} Js$ Reduced Plank constant, also called Dirac Constant
$m_e = 9.1x10^{-31}Kg$ Mass of an Electron
$h = 6.62x10^{-34}Js$
$m_p = 1.6726219x10^{-27}Kg$ mass of a proton
$m_n = 1.674927x10^{-27}Kg$ mass of a neutron

\colorbox{BurntOrange}{Quantum Mech in 1D}
$\langle x \rangle =$ Average position of a particle
$\langle p \rangle =$ Average momentum of a particle
$\langle Q \rangle =$Operators in quantum mechanics
$\langle U \rangle =$ Average potential energy
$\langle K \rangle =$ Average kinetic energy
$\langle E \rangle = \langle K \rangle + \langle U \rangle $ total energy for a particle



If a measurement of position is made with precision $\Delta x$ and a simultaneous measurement of momentum in the x direction is made with precision $\Delta p_x$, then the product of the two uncertainties can never be smaller than $\frac{\hbar}{2}$

\hl{Quantum Mech in 1D}\\
\hl{Tunneling}\\
\hl{Quantum Mech in 3D}\\
Rydberg Energy = $\frac{Ke^2}{2a_0} = 13.6$ eV\\
The Bohr radius = $a_0 = \frac{\hbar}{m_eKe^2}$\\
Bohr magneton = $\mu_B = \frac{e\hbar}{2m} = 9.274x10^{-24} J/T$\\
\hl{Atomic Structure}\\



Modern Physics Constants\\
Weins Displacement  = $\lambda_{max}T = 2.898x10^{-3} mK$\\
Stefans-Boltzmann constant= $\sigma = 5.67x10^{-8} Wm{-2}K^{-4}$\\
Gravitational Constant = $ G =  6.674x10^{-34}m^2Kgs^{-1}$\\
Planks Constant $h=6.626x10^{-34}m^2kgs^{-1}$\\
Mass of Electron $m_e=9.109x10^{-31}kg$\\
Charge of Electron $q_e = -1.602x10^{-19}C$\\
Charge of Proton $q_p = 1.602x10^{-19}C$\\
Elementary charge $e= 1.602x10^{-19}C$\\
mass of proton $m_p=1.6727x10^{-27}kg$\\
mass of neutron $m_n=1.6727x10^{-27}kg$\\
mass of electron $m_e=9.109x10^{-31}kg$\\
Rydberg Constant $R_{\infty}= \frac{m_e e^4}{8\epsilon_0^2 h^3 c} = 1.097x10^{7} m^{-1}$\\
Bohr radius $a_0 =\frac{4\pi \epsilon_0 \hbar^2}{m_e e^2} = 5.292x10^{-11}m$\\
permittivity of free space $\epsilon_0 = 8.854x10^{-12} s^4A^2m^{-3}kg^{-1}$\\
reduced plank constant  $\hbar = 1.054x10^{-34} J s$\\
coulomb constant  $k_e = \frac{1}{4\pi \epsilon_0} = 8.987x10^9 Nm^2C^{-2}$\\ 
electron orbital filling levels\\
\[
1s_{2}^{2} 
2s_{4}^{2} 
2p_{10}^6 
3s_{12}^{2}
3p_{18}^{6}
4s_{20}^2
3d_{30}^{10}
4p_{36}^{6}
5s_{38}^{2}
\]
\[
4d_{48}^{10}
5p_{54}^{6}
6s_{56}^{2}
4f_{70}^{14}
5d_{80}^{10}
6p_{86}^{6}
7s_{88}^{2}
5f_{102}^{14}
6d_{112}^{10}
7p_{118}^{6}
\]






%\end{multicols}


\end{document}  