\documentclass[6pt, oneside]{article}   	% use "amsart" instead of "article" for AMSLaTeX format
\usepackage[margin=0.25in]{geometry}                		% See geometry.pdf to learn the layout options. There are lots.
%\geometry{letterpaper}                   		% ... or a4paper or a5paper or ... 
\geometry{landscape}                		% Activate for for rotated page geometry
%\usepackage[parfill]{parskip}    		% Activate to begin paragraphs with an empty line rather than an indent
\usepackage{graphicx}				% Use pdf, png, jpg, or eps§ with pdflatex; use eps in DVI mode
								% TeX will automatically convert eps --> pdf in pdflatex		



\usepackage[utf8]{inputenc}
\usepackage[english]{babel}
 
\usepackage[usenames, dvipsnames]{color}
 

 
\usepackage{multicol}
\usepackage{color,soul}

\title{Modern Physics Formulas}
\author{Joseph Crandall}
%\date{}							% Activate to display a given date or no date

\begin{document}
%\begin{multicols}{1}

%\maketitle
\colorbox{YellowGreen}{Joe Crandall's ModernPhysics Formula Sheet, covers special relativity and basic quantum mechanics, equations are repeated, }
\colorbox{SeaGreen}{Fields of Study}
\colorbox{Orange}{Section}
\colorbox{Red}{Topic unknown}
\colorbox{Cyan}{Topic}
\colorbox{Violet}{SubTopic}
\colorbox{Yellow}{Spacing for readability}
\colorbox{SeaGreen}{Special Relativity}
\colorbox{Orange}{Principal of Relativity}
\colorbox{Orange}{ The Nature of Time}
\hl{$\Delta t >  \Delta s > \Delta \tau$}
\colorbox{Orange}{The Metric Equation}
$\Delta s^{2} = \Delta t^2 - \Delta d^2 = \Delta t^{2} - \Delta x^{2} - \Delta y^{2} - \Delta z^{2}$
\colorbox{Orange}{Proper Time}
\hl{$\Delta \tau_{AB} = \int_{t_A}^{t_B} (1-v^2)^{1/2} dt$}
$ \Delta \tau_{AB} = (1-v^2)^{1/2} \Delta t_{AB} $
\hl{$v = constant $}
$ v_0 = \frac{\sqrt{Gm}}{r} $
\hl{$ G = 6.673 x 10^{-11}$}
Binomial\,Approximation $ \sqrt{1-v^2} \approx 1-\frac{1}{2}v^2 for\, v \ll c$
\hl{$ \Delta \tau_{AB} - \Delta t_{AB} = -\frac{1}{2}v^2 \Delta t_{AB}$}
\colorbox{Orange}{Coordinate Transformations}
$\Delta t^{\prime} = \gamma(\Delta t-\beta \Delta x) $
\hl{$\Delta x^{\prime} = \gamma(-\beta \Delta t + \Delta x) $}
$\Delta y^{\prime} = \Delta y $
\hl{$\Delta z^{\prime} = \Delta z $}
$\Delta t = \gamma(\Delta t^{\prime} + \beta \Delta x^{\prime}) $ 
\hl{$\Delta x = \gamma(\beta \Delta t^{\prime} + \Delta x^{\prime}) $ }
$\Delta y = \Delta y^{\prime} $
\hl{$\Delta z = \Delta z^{\prime} $}
$\gamma = 1/\sqrt{1-\beta^{2}} $
\colorbox{Orange}{Lorentz Contraction}
$L = L_R \sqrt{1-v^2} $
\hl{$\theta = tan^{-1} (\frac{tan\theta^{\prime}}{\sqrt{1-\beta^2}})$}
\colorbox{Orange}{The Cosmic-Speed Limit}
$v_x^{\prime} = \frac{v_x - \beta}{1 - \beta v_x}  $
\hl{$v_x = \frac{v_x^{\prime} + \beta}{1 + \beta v_x^{\prime}}  $}
$v_y^{\prime} = \frac{v_y \sqrt{1 - \beta^2}}		{1 - \beta v_x}  $
\hl{$v_y= \frac{v_y^{\prime} \sqrt{ 1 - \beta^2}}		{1 + \beta v_x^{\prime}} $}
$v_z^{\prime} = \frac{v_z \sqrt{1 - \beta^2}}		{1 - \beta v_x}  $
\hl{$v_z = \frac	{v_z^{\prime} \sqrt{ 1 - \beta^2}}	{1 + \beta v_x^{\prime}} $}
\colorbox{Orange}{Principal of Four-Momentum / Conservation of Four-Momentum}
\[
  \left[ {\begin{array}{c}
   P_t\\
   P_x\\
   P_y\\
   P_z\\
  \end{array} } \right]
  =
   \left[ {\begin{array}{c}
   m / \sqrt{1-v^2}\\
   mv_x / \sqrt{1-v^2}\\
   mv_y / \sqrt{1-v^2}\\
   mv_z / \sqrt{1-v^2}\\
  \end{array} } \right]
\quad \quad \quad
   \left[ {\begin{array}{c}
   E_1\\
   P_1x\\
   P_1y\\
   P_1z\\
  \end{array} } \right]
  +
    \left[ {\begin{array}{c}
   E_2\\
   P_2x\\
   P_2y\\
   P_2z\\
  \end{array} } \right]
  =
    \left[ {\begin{array}{c}
   E_3\\
   P_3x\\
   P_3y\\
   P_3z\\
  \end{array} } \right]
  +
    \left[ {\begin{array}{c}
   E_4\\
   P_4x\\
   P_4y\\
   P_4z\\
  \end{array} } \right]
\]

\hl{$E = \frac{m}{\sqrt{1-v^2}} $}
$\rho  = \frac{mv}{\sqrt{1-v^2}} $
\hl{$K  = E - m $}
$\frac{\rho}{E} = v$
\hl{Four Momemtum's Time and Space Components}
$\rho_t = \frac{m}{\sqrt{1-v^2}} $
\hl{$\rho = m + \frac{1}{2}mv^2\,when\,v\ll 1 $}
$K = m(\frac{1}{\sqrt{1-v^2}}-1) $
\hl{$E_{rest} = mc^2$}
Proporties of Four Momentum Conversion
\hl{$P_t^{\prime} = \gamma(P_t - \beta P_x) $}
$P_x^{\prime} = \gamma(-\beta P_t + P_x) $ 
\hl{$Four Momentum of Light$}
$\frac{\rho}{E} = v = 1 \qquad \rho = E \qquad$
\hl{$m^2 = E^2 - \rho^2 = 0$}

\colorbox{SeaGreen}{Quantum}
$E = \frac{1}{2}mv^2$\\
Binomial approximation, is x is a real number close to 0 and alpha is a real number then $(1+x)^{\alpha}\approx1+\alpha x$\\ 
\colorbox{BurntOrange}{The Quantum Theory of Light}
\colorbox{Cyan}{BlackBody Radiation}
\hl{Stefan's Law $e_{total} = a\sigma T^4$}
Wien's Displacement law  $\lambda_{max}T = 2.898x10^{-3} mK$
\hl{Plank Black Body Radiation Formula $u(f,T)=\frac{8\pi hf^3}{c^8}(\frac{1}{e^{\frac{hf}{k_bT}}-1})$}
\colorbox{Cyan}{Photoelectric Effect}
Einstein's theory of the photoelectric effect $K_{max}=hf-\phi$
\hl{Photon Energy $E = \frac{hc}{\lambda} = hf \quad f =\frac{c}{\lambda}$}
Bragg equation $n\lambda = 2dsin\theta \quad n = 1,2,3,...$
\hl{X-ray photon emission $\lambda_{min} = \frac{hc}{eV}$}
If you increase the intensity of the light you increase the measured current
\colorbox{Cyan}{Compton Effec}
\hl{$\lambda^\prime - \lambda_0 = \frac{h}{m_ec}(1-cos\theta)$}
Energy Conservation $E = m_ec^2 = E^{\prime} + E_{e}$
\hl{Momentum Conservation $p = p^{\prime}cos\theta + p_e cos\phi$}
Energy and Momemtm $p_{photon} = \frac{E}{c} = \frac{hf}{c} = \frac{h}{\lambda}$
\hl{$p_{electron} = mv$}

\colorbox{BurntOrange}{The Particle Nature of Matter}
$A = \frac{4}{3}\pi r^2$
\hl{$f=\frac{E_i - E_f}{h}$}
$c = fh$
\hl{$\lambda = \frac{ch}{E_i - E_f}$}
$K = \frac{1}{2}mv^2$ 
\hl{$mv = \frac{E_{photon}}{c}$}
$E= E_{photon} + \frac{1}{2}mv^2$
\hl{$p = \frac{E_{photon}}{c}$}
$\frac{1}{\lambda}=R(\frac{1}{n_f^2}-\frac{1}{n_i^2})$
\hl{$f = \frac{c}{\lambda}$}
$E = hf$
\hl{$\frac{E}{c} = p$}
$E = hcR \big( \frac{1}{n_f^2} - \frac{1}{n_i^2} \big)$ if $n_i > n_f$ energy released
\hl{$\frac{f}{c} = \frac{1}{\lambda}$}
$f = \frac{E_i - E_f}{h}$
$\lambda = \frac{ch}{E_i - E_f}$
\hl{$\lambda = \frac{ch}{E_i - E_f}$}
\colorbox{Red}{Millikan's stuff}
\colorbox{Cyan}{The Bohr Atom}
\hl{$\frac{1}{\lambda}=R_{\infty}(\frac{1}{n_f^2-n_i^2})$}
Plank-Enstein formula $E_i - E_f = hf$
\hl{size of allowed electron orbits $m_evr = n\hbar \quad \hbar = \frac{h}{2\pi}$}
n = orbital levels
\hl{Total energy of the atom $E = K + U = \frac{1}{2}m_ev^2 - k\frac{e^2}{r}$}
permittivity of free space $\epsilon_0 = 8.854x10^{-12} s^4A^2m^{-3}kg^{-1}$
\hl{Radii of Bohr orbits in hydrogen $r_n=\frac{n^2\hbar^2}{m_eke^2} \quad n = 1,2,3,..$}
Bohr radius the smallest radius occurs for n = 1, is called the Bohr radius, denoted by $a_0$
\hl{$a_0 =\frac{4\pi \epsilon_0 \hbar^2}{m_e e^2} = 5.292x10^{-11}m$}
Energy levels of hydrogen and hydrogen like $E_n = -\frac{ke^2}{2a_0}(\frac{Z^2}{n^2})  = -13.6\frac{Z^2}{n^2}eV$ n = 1,2,3... z = 1,2,3...
\colorbox{Cyan}{Emission wavelengths of hydrogen}
\hl{$\frac{1}{\lambda} = \frac{f}{c} = \frac{ke^2}{2a_0hc}(\frac{1}{n_f^2} - \frac{1}{n_i^2}))$}
{could the 1's be replaced by Z squared}
\colorbox{Red}{Corespondanced Principal}

\colorbox{BurntOrange}{The Wave Nature of Matter}
\hl{$p=mv$}
$\lambda = \frac{h}{p}$
\hl{$p = \gamma mv$}
$\gamma = (1-\frac{v^2}{c^2})^{-\frac{1}{2}}$
\hl{$\lambda = \frac{h \sqrt{1-(\frac{v}{c})^2}}{mv}$}
$\lambda = \frac{h}{p}$
\hl{$p = \frac{h}{\lambda}$}
$k = \frac{p^2}{2m}$
\hl{$k = \frac{3}{2}K_bT$}
$\lambda p = \frac{hc}{E_{photon}}$
\hl{$E = fh$}
$f = \frac{c}{\lambda}$
\hl{$E = \frac{ch}{\lambda}$}
$\lambda = \frac{h}{p}$
\hl{$KE = \frac{1}{2}mv^2$}
$p=mv$
\hl{$v=\frac{p}{m}$}
$E=cp$
\hl{$KE = \frac{1}{2} \frac{p^2}{m}$}
$KE = \frac{1}{2}\frac{E^2}{c^2 m}$
\hl{$v = \frac{d}{t}$}
$\lambda = \frac{h}{p}$
\hl{$p = mv$}
$\lambda = \frac{h}{mv}$
\hl{$\lambda = \frac{h}{p}$}
$p = mv$
\hl{$E = \frac{1}{2}mv^2$}
$E = \frac{1}{2}pv$
\hl{$p=\sqrt{2E}$}
$\lambda = \frac{h}{\sqrt{k2m}}$
\hl{$k = \frac{p^2}{2m}$}
$k = (\frac{h}{\lambda})^2 \frac{1}{2me}$
\hl{$KE\leftrightarrow E\leftrightarrow k \leftrightarrow K$ for most of these problems}
$\Delta p_x \Delta x \geq \frac{\hbar}{2}$ 
\hl{$p_x = \frac{\hbar}{2x} = \frac{h}{4\pi x}$}
$E = \frac{1}{2}\frac{p^2}{m}$
\hl{$p = \sqrt{2Em}$}
$\sqrt{2Em} = \frac{h}{4\pi x}$
\hl{$2Em = ( \frac{h}{4 \pi x} )^2$}
$E = \frac{( \frac{h}{4\pi x})^2}{2m}$
\hl{$E = cp$}
$p_x = \frac{h}{4\pi x}$
\hl{$\frac{E}{c} = \frac{h}{4\pi x}$}
$E = \frac{ch}{4\pi x}$
\hl{$\Delta E \Delta t \approx \frac{\hbar}{2}$}
$\Delta p_x \Delta x \geq \frac{\hbar}{2}$
\hl{$\Delta E = \frac{h}{4\pi \Delta t}$}
$E = cp$
\hl{$p = \frac{E}{c}$}
$\lambda = \frac{hc}{E} $


\colorbox{Cyan}{Pilot Waves of De Broglie}
\hl{De Broglie Wavelengths $\lambda = \frac{h}{p} \quad f = \frac{E}{h} \quad p = \gamma mv \quad \gamma = (1-\frac{v^2}{c^2})^{-\frac{1}{2}}$}
$E^2 = p^2c^2 + m^2c^4 = \gamma^2m^2c^4$
\hl{Bohr orbits arise because the electron matter waves interfere constructively of a circular orbit}
$n\lambda = 2\pi r \quad \lambda = \frac{h}{m_ev}$
\hl{$m_evr = n\hbar$}
\colorbox{Cyan}{Heisenberg Uncertainty Principle}
$\Delta x \Delta p_x \geq \frac{\hbar}{2}$
\hl{$\Delta y \Delta p_y \geq \frac{\hbar}{2}$}
$\Delta z \Delta p_z \geq \frac{\hbar}{2}$
\hl{$\Delta t \Delta E \geq \frac{\hbar}{2}$}

\colorbox{BurntOrange}{Quantum Mechanics in One Dimension}
\hl{$k =\frac{n\pi}{L}$}
$E_n=n^2 E_1$
\hl{$E_1=\frac{\pi^2 \hbar^2}{2mL^2}$}
$E=\frac{hc}{\lambda}$
\hl{$n_i^2(E_1) - n_f^2E_1 = \Delta E$}
$n=\frac{(\frac{\Delta E}{E_1} + 1)}{2} = \frac{(\frac{\frac{hc}{\lambda}}{E_1}-1)}{2}$
\hl{$hf = E_i - E_f$}
$\frac{h}{p}$
\hl{$E_n =\frac{ n^2 \pi^2 \hbar^2}{2mL^2}$}
$n = \frac{2Lmv}{h}$
\hl{$p=\frac{h}{\lambda}$}
$\Delta E = \frac{hc}{\lambda} = hf$
\hl{$\frac{hc}{\lambda} = E_i = E_f$}
$\Delta E  = \frac{n^2 \pi ^2 \hbar^2}{2mL^2}$
\hl{$\frac{hc}{\lambda} = \frac{1}{L^2}(\frac{n_i^2 \pi^2 h^2}{4\pi^2 2 m} - \frac{n_f^2 \pi^2 h^2}{4\pi^2 2 m})$}
$L = \sqrt{\frac{\lambda}{hc}(\frac{n_i^2 \pi^2 h^2}{4\pi^2 2 m} - \frac{n_f^2 \pi^2 h^2}{4\pi^2 2 m})}$
\hl{$\lambda = \frac{hc}{E_i = E_f}$}
$E = \frac{n^2 h^2}{8mL^2}$
\hl{$\lambda = \frac{hc 2mL^2}{(n_i^2 h^2)-(n_f^2 h^2)}$}
$E = \frac{h^2}{8mL^2}$
\hl{If halfed $L_{new} = \frac{L_old}{2}$}
Wavenumber = k
\hl{angular freqeuncy = $\omega$}
$kL = n\pi$
\hl{$k = \frac{2\pi}{\lambda}$}
$\lambda = \frac{h}{p}$
\hl{$p = mv$}
$n=\frac{L(\frac{2\pi}{\frac{h}{mv}})}{\pi}$
\hl{$n = \frac{2L mv}{\pi}$}
$k = \frac{p}{\hbar}$
\hl{$\omega = \frac{E}{\hbar}$}
for nonrelativistic particle $\omega(k) = \frac{\hbar k^2}{2m}$
\hl{$E_n = \frac{n^2 h^2}{8mL^2}$}
$n = \frac{2Lmv}{h}$
\colorbox{Cyan}{Corespondance principle for electrons}
\hl{$p = \frac{x}{L}-\frac{1}{2\pi n} sin(\frac{2\pi n x}{L})$}
classically left hand wall $\frac{x}{L}$
\colorbox{Cyan}{Probabilities of Infinite Square Well}
\hl{$p = \frac{x}{L}-\frac{1}{2\pi n} sin(\frac{2\pi n x}{L})$}
\colorbox{Cyan}{Normalization of Wave Functions}
$\psi = Asin(X)$
\hl{$1 = \int_0^{\pi} |Asin(x)|^2 dx$}
$A = \sqrt{\frac{2}{\pi}}$
\colorbox{Red}{Particle Probability}
\colorbox{Cyan}{Penetration Depth}
\hl{probability finding particle $ x = L+\eta$ to $x=\eta$ $\eta=$ penetration distance L=width of wall $\langle p \rangle = \frac{1}{e^2}$}
how many $\eta$ outside well for outside/inside ratio = x $\frac{ln(x)}{2} = \eta$
\colorbox{Cyan}{Penetration at Top/Bottom of Well}
\hl{$E_n = \frac{n^2 \pi \hbar^2}{2m(L+2\delta)^2}$}
$\delta = \frac{\hbar}{\sqrt{2m(U-E)}} = \frac{1}{\alpha}$ U = depth of the well E = energy of the electron
\colorbox{Red}{Energy levels in finite square well}
\colorbox{Red}{nucleaus as square well potential}
\colorbox{Cyan}{Transitions in a Harmonic Oscillator}
\hl{$\lambda = \frac{h}{p}$}
$k = \frac{2\pi}{\lambda}$
\hl{$E = \frac{hc}{\lambda}$}
$E_n = (n+frac{1}{2})\hbar \omega$
\hl{$\Delta E = \hbar \omega$}
$E = \frac{hc}{\lambda}$
\hl{$f = \frac{c}{\lambda}$}
$E_i - E_f = hf$
\hl{$\omega = \frac{hc}{\lambda \hbar ((n_i + 1/2)-(n_f + 1/2))}$}
$\omega = \frac{2\pi c}{\lambda ((n_i + 1/2)-(n_f + 1/2))}$
\hl{$\Delta E = \frac{hc}{\lambda}$}
$\Delta E = \frac{h\omega ((n_i + 1/2)-(n_f + 1/2))}{2\pi}$
\hl{$\lambda = \frac{2\pi c}{\omega ((n_i + 1/2)-(n_f + 1/2)) }$}
\colorbox{Cyan}{Energy Levels in harmonic oscilator}
spring constant = k
\hl{harmonic oscillator zero point energy n = 0 $E_0 = \frac{1}{2} \frac{h}{2\pi} \sqrt{\frac{k}{m}}$}
$\Delta E = \hbar \omega $
\hl{$hf = \hbar \omega = E$}
$\omega = \sqrt{\frac{k}{m}}$
\hl{$E_n = (n+\frac{1}{2})(\frac{h}{2\pi})(\sqrt{\frac{k}{m}})$}
$E = \frac{hc}{\lambda}$
\hl{$\frac{hc}{\lambda} = \hbar \omega((n_i + 1/2)-(n_f + 1/2))$}
$\omega = \sqrt{\frac{k}{m}}$
\hl{$\lambda = \frac{hc\sqrt{m}2\pi}{h\sqrt{k}((n_i + 1/2)-(n_f + 1/2))}$}
\colorbox{Cyan}{Expectation values in an Infinite Square well}
$n = 1 \langle x \rangle = \frac{L}{2}$
\hl{$n = 1 \langle x^2 \rangle = 0.283 L^2$}
$n = 1 \langle p \rangle = 0$
\colorbox{Cyan}{Expectation Value and Most Probable Location}
\hl{$\psi(x) = Ae^{-x}(1-e^{-x})$for$x\geq0$ and $\psi(x) = 0 $for$x\leq0$}
$\int_0^L (\psi_n(x))^2 dx = 1$
\hl{$\int_0^L (Ae^{-x}(1-e^{-x}))^2 dx = 1$}
$ A^2 \int_0^L e^{-4x} - 2e^{-3x} + e^{-2x} dx = 1$
\hl{$ (\frac{-e^{-4L}}{4})-(\frac{-1}{4}) + (\frac{2e^{-3L}}{3})-(\frac{2}{3}) + (\frac{-e^{-2L}}{2})-(\frac{-1}{2}) = \frac{1}{A^2}$}
$lim \rightarrow \infty$
\hl{$A = \sqrt{12}$}
$A = 3.464$
\colorbox{Fuchsia}{electron most likeley to be found}
\hl{$(\psi_n)^2 dx = 0$}
$\psi_n = 3.464 e^{-x}(1-e^{-x})$
\hl{$(3.464 e^{-x}(1-e^{-x}))^2 dx = 0$}
...
\hl{$11.9993(\frac{2(1-e^{-x})}{e^{3x}} - \frac{2(1-e^{-x})^2}{e^{2x}})  = 0$}
...
\hl{$x = 0.693147$}
\colorbox{Fuchsia}{calculate expectation value of position x}
$\langle x \rangle = \int_0^{\infty} x|\psi(x)|^2 dx$
\hl{$\langle x \rangle = \int_0^{\infty} x| 3.464 e^{-x} (1-e^{-x})) |^2 dx = 0$}
$\langle x \rangle = 12 \int_0^{\infty} x (e^{-x} - e^{-2x})^2 dx = 0$
\hl{$=\frac{13}{12} = 1.0833$}

\colorbox{Thistle}{Quantum Mechanics in One Dimension, from test 2}
\colorbox{Red}{WaveFunction}
\colorbox{Red}{Classic Probability}
\colorbox{Cyan}{Schrodiner Equation}
\hl{$E\psi(x) = U(x)\psi(x) - \frac{\hbar^2}{2m}\frac{d^2\psi}{dx^2}$}
$E_n = \frac{\hbar^2k^2}{2m} = \frac{n^2\pi^2\hbar^2}{2mL^2}$
\colorbox{Cyan}{Particle in a Box}
\colorbox{Violet}{Find the Wave Function}
\hl{$\psi(x) = Asinkx + B coskx$}
Boundary conditions dictates that $\phi(x) = 0$ at x = 0 and x = L
\hl{for x = 0, this means that B = 0}
for x = L, this means that $kL = n\pi$ fro integer vaues of n
\hl{$\psi_n(x) = A sin(\frac{n\pi x}{L})$}
Find normalization constant for A
\hl{$\int_{\infty}^{\infty} \psi_n(x)\psi_n(x)  dx = 1 \rightarrow $}
$A^2 \int_{0}^{L}  sin^2(\frac{n\pi x}{L}) \rightarrow A = \sqrt{\frac{2}{L}}$
\hl{Normalized wave function looks like th case of standing waves on a string with fixed ends:}
$\psi_n(x) = \sqrt{\frac{2}{L}} sin(\frac{n\pi x}{L})$
\colorbox{Cyan}{Find the Energy Levels}
\hl{From  boundary conditions discrete wave number is:}
$k_n = \frac{n\pi}{L}=\sqrt{\frac{2mE_n}{\hbar^2}}$
\hl{This yields discrete energies}
$E_n= \frac{\hbar^2 k_n^2}{2m} = \frac{\hbar^2}{2m}(\frac{n^2\pi^2}{L^2})=\frac{n^2h^2}{8mL^2}$
\colorbox{Cyan}{Particle in a box:Classical vs Quantum}
\hl{Classical Probability}
$P(x)=\frac{1}{L}$
\hl{Classical particle moving in a valley}
$P_{Cl}(x) = \frac{2}{T}\frac{1}{v(x)}=\frac{2}{T}\sqrt{\frac{m}{2(E-U(x))}}$
\hl{Quantum probability}
$P(x) = |\psi(x)|^2 = \frac{2}{L}sin^2(\frac{n\pi x}{L})$
\hl{KE of particle in Box}
$E_n = \frac{1}{2}mv^2 = \frac{n^2h^2}{8mL^2} \quad E_{min} = \frac{p^2}{2m} = \frac{\hbar^2}{8mL^2}$
\colorbox{Cyan}{Finite Square Well}
\hl{Schrodinger eqn outside finite well:}
$\alpha^2 = \frac{2m(V_0 - E)}{\hbar^2} \quad k^2 = \frac{2mE}{\hbar^2}$
\colorbox{Cyan}{Penetration Depth}
\hl{$\eta = \frac{1}{\alpha} = \frac{\hbar}{\sqrt{2m(V_0 - E)}}$}
\colorbox{Cyan}{Harmonic Oscillator}
Potential energy
\hl{$U(x) = \frac{1}{2}kx^2$}
Energy Levels
\hl{$E_n = (n+\frac{1}{2})\hbar \sqrt{\frac{k}{m}} = (n+\frac{1}{2})\hbar \omega$}
$hf = \hbar\omega$
\hl{Zero point energy}
$E_0 = \frac{1}{2}\hbar \omega$

\colorbox{BurntOrange}{Tunneling Phenomena}
\colorbox{Cyan}{Tunneling Phenomena}
\colorbox{Cyan}{Potential Step}
\hl{Reflection coefficient }
$R = (\frac{k-\alpha}{k+\alpha})^2$\\ $\alpha = k_2$ $k = k_1$
\hl{$k = \frac{p}{\hbar}$  $\omega = \frac{E}{\hbar}$ $k=\frac{\omega p}{E}$}
Transmission coefficient
\hl{$T(E)=\{1+\frac{1}{4}[\frac{U^2}{E(U-E)}]sinh^2\alpha L \}^{-1}$}
$T = 1-R = \frac{4k\alpha}{(k+\alpha)^2}$
\hl{Quantization o Angular Momentum}
$L=\sqrt{l(l+1)}$
\hl{Probability Distribution}
\colorbox{Cyan}{what is Bohr radius}
$r = \frac{n^2 a_0}{Z}$
\hl{$T = 1 - R$}
$R = (\frac{k-\alpha}{k+\alpha})^2$
\hl{$T(E)=\{1+\frac{1}{4}[\frac{U^2}{E(U-E)}]sinh^2\alpha L \}^{-1}$}
$sinh (x) = \frac{e^{x} + e^{-x}}{2}$
\hl{$T \approx e^{-2\alpha L}$}
\colorbox{Cyan}{Reflection and Transmision Coeficients for a potential Barrier}
\hl{Case1) $E<V_0$ and $\alpha^2 = \frac{2m(V_0-E)}{\hbar^2}> 0$}
Case2) $E>V_0$ and $\alpha^2 = \frac{2m(E-V_0)}{\hbar^2}> 0$

\colorbox{BurntOrange}{Quantum Mechanics in Three Dimensions}
\colorbox{Cyan}{article in a 3D Box}
\hl{Normalized Wavefunction $\sqrt{\frac{8}{L^3}}sin(\frac{n_x\pi x}{L})sin(\frac{n_y\pi x}{L})sin(\frac{n_z\pi x}{L})$}
Quantized Energy $E = (n_x^2 +n_y^2 +n_z^2)\frac{\pi^2 \hbar^2}{2mL^2}$
\hl{Schrodinger equation in three dimensions $-\frac{\hbar^2}{2m}\nabla^2\psi + U(r)\psi = i\hbar \frac{\partial\psi}{\partial t}$}
Laplacian = $\nabla^2 = \frac{\partial^2}{\partial x^2}+\frac{\partial^2}{\partial y^2}+\frac{\partial^2}{\partial z^2}$
\hl{The time-independent Schrodinger equation $-\frac{\hbar^2}{2m}\nabla^2\psi + U(r)\psi = E\psi (r)$}  
allowed values of momentum components for particle in a box 
\hl{$|p_x| = \hbar k_1 = n_1\frac{\pi \hbar}{L}$ for $ n_1 = 1,2,... $}
$|p_y| = \hbar k_2 = n_2\frac{\pi \hbar}{L}$ for $ n_2 = 1,2,... $
\hl{$|p_z| = \hbar k_3 = n_3\frac{\pi \hbar}{L}$ for $ n_3 = 1,2,... $}
Discrete energies allowed for a particle in a box
\hl{$E=\frac{1}{2m}(|p_x|^2+|p_y|^2+|p_z|^2)=\frac{\pi^2\hbar^2}{2mL^2} \{n_1^2+n_2^2+n_3^2\}$}
$E=\frac{1}{2m}(|p_x|^2+|p_y|^2+|p_z|^2)=\frac{\pi^2\hbar^2}{2m} \{(\frac{n_1}{L_1})^2+(\frac{n_2}{L_2})^2+(\frac{n_3}{L_3})^2\}$
\hl{Hydrogen Atom and Hydrogen Like}
$E_n = \frac{-13.6}{n^2}eV$
\hl{Degeneracy = $n^2$}
$E_n=\frac{-Ke^2}{2a_0}\{\frac{Z^2}{n^2}\}$
\hl{Angular Momentum Vector = $|L|=\sqrt{l(l+1)}\hbar$}
Z component = $L_z=m_l\hbar$
\hl{Orientations of L are quantized = $cos\theta = \frac{L_z}{|L|}=\frac{m_l}{\sqrt{l(l+1)}}$}
Change in n and emission of Photon $13.6eV(\frac{1}{n_2^2}-\frac{1}{n_1^2})=\frac{hc}{\lambda}(\frac{1}{1.6x10^{-19}})$
\colorbox{Violet}{expectation values} Example expectation values for a particle in a box in a general state labeled by n
$\langle x \rangle= \frac{L}{2}$
\hl{Hydrogen ground state most probable radius(and hydrogen like)}
$P(r)=\frac{4Z^3}{a_0^3}r^2e^{\frac{-2Zr}{a_0}}$
\hl{average radius of the electron $r = \frac{a_0}{Z}$}
fixed n, electron with highest angular momentum found at distance $r = n^2 a_0$
\hl{the letters s,p,d,f used to designante $l = 0,1,2,3$}
\colorbox{Blue}{Zeeman Effect}
magnetic moment $\mu = IA$ Potential energy due to external magnetic field in z-direction $U=-\vec{\mu}B = \mu_B m_l B$
\hl{magnet placed in B field has interaftion potential energy $ U = -\mu B$}
magnetic moment due to current loop $\vec{\mu} =-\frac{e}{2m}\vec{L}$
\hl{Zeeman Effect $E = E_n + \mu_B m_l B$}
\colorbox{Blue}{electron spin}
intrinsic $S = \frac{\sqrt{3}}{2}\hbar$
\hl{angle between electron's spin angular momentum and its +z component}
$Cos \theta = \frac{S_z}{S}=\frac{\frac{\hbar}{2}}{[\sqrt{3/2\hbar}]}$ = $\frac{1}{\sqrt{3}}$
\hl{Electron transitions with spin $(2n+1)$}
degeneracy with spin $2(n^2)$
\hl{electron spin}

\colorbox{Orange}{Atomic Structure}
J is total angular momentum with spin $J = L + S$
\hl{high l = circular orbit, low l - eleptical orbit due to pull from nucleaus}
$E_n = \frac{-13.6Z^2}{n^2}eV$
\hl{effective nuclear charge $Z_{eff}$}
$E_{nl}=-\frac{Z_{eff}^{2}}{n^2}(13.6eV)$
\hl{Mosely law for xray frequency}
$f=(2.47x10^{15}Hz)(Z-1)^2$

%\end{multicols}
\end{document}  